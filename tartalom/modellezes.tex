\chapter{Modellezés}
A probléma négy lényeges részből áll, amelyeket többé-kevésbé külön lehet kezeni egymástól:
\begin{enumerate}
	\item Balun
	\item Differenciális vonal
	\item Antenna
	\item Földkitöltésben gerjesztett áramokat csökkentő mintázat
\end{enumerate}
\par Ezek fejlesztése több szempontból is párhuzamosan kellett, hogy történjen, mert sem a balun, sem az antenna nem valósítható meg széles paramétertartományban. Az egyik lényeges korlátozó paraméter a karakterisztikus impedancia -- az antenna bemeneti impedanciáját a balun kimeneti impedanciájához kell illeszteni a kérdéses frekvenciatartományban (\SIrange{2405}{2485}{MHz}) és természetesen ehhez az impedanciához kell méretezni az antennát tápláló differenciális vonal karakterisztikus impedanciáját, hogy ez se okozzon lényeges reflexiót.
\section{Differenciális vonal}
Mivel az antenna differenciális táplálású, érdemes differenciális tápvonallal gerjeszteni. Az antenna által lefedett terület alatt nem szabad földkitöltést használni az alsóbb rézrétegeken, mert ezek erősen lerontanák az antenna tulajdonságait. Így a tápláló differenciális vonal is célszerűen olyan, hogy nincs alatta földkitöltés. Ez nem feltétlenül kell, hogy így legyen, de ehhez a verzióhoz ragaszkodtam a félévben. Egy ilyen tápvonal a CPS (coplanar strip). A tápvonal keresztmetszete \aref{fig:cps}. ábrán látható.
\begin{figure}[h]
	\centering
	\includegraphics[width=0.6\textwidth]{kep/cps.pdf}
	\caption{Az antennát tápláló szimmetrikus koplanár vonal (CPS) keresztmetszete.}
	\label{fig:cps}
\end{figure}
\par Sajnos az ilyen tápvonalakban nem lehetséges a tisztán TEM hullámterjedés, és diszperzió lép fel. Ennek oka, hogy az erővonalak egy jelentős része a levegőn keresztül záródik, egy másik nagy részük pedig a szubsztráton keresztül (esetleg szubsztrát-levegő-szubsztrát sorozaton keresztül), emiatt a haladó hullám különböző részei különböző effektív dielektromos állandót és ezáltal terjedési együtthatót tapasztalnak. Az ilyen vonalakat kvázi-TEM közelítéssel érdemes kezelni, de ez pontatlanságokat visz a számításba az elhanyagolások mértékétől függően, ami tápvonalspecifikus.
\par A CPS erősebb diszperzív volta miatt ritkábban használják, mint például a CPW(G) tápvonalat. A CPS-t leíró közelítő formulák is kevésbé elterjedtek \cite{coupled-slots, analytical}, nagyobb egyszerűsítésekkel adódnak és emiatt ez a tápvonal nem is szerepel az ismertebb impedanciakalkulátor programokban, mint például a KiCAD, vagy az AWR MWO programcsomagokhoz tartozó, vagy internetes kalkulátorokban. A tápvonal karakterisztikus impedanciáját a CST ,,Waveguide Port'' funkcióját használva becsültem meg. Ilyen port használatakor a szimuláció során kiszámolódnak különböző módusok a porthoz (ami a CPS keresztmetszetéhez illeszkedik) és egy meghatározott alap módusra adódó karakterisztikus impedanciával számol tovább a program. Az így adódó értéket használtam fel, mint közelítő karakterisztikus impedancia értéket.
\par Az említett nehézségek miatt ezt a pontatlanságot elfogadtam, ezzel együtt dolgoztam. Ezt a becslést az is alátámasztja, hogy egy hasonló felépítésű tápvonal, a slotline impedanciájával alulról becsülhető az azonos hézaggal és szubsztráttal elkészített CPS impedanciája. A slotline keresztmetszete a CPS-ével megegyező, azzal a kivétellel, hogy a két vezető oldalirányban sokkal nagyobb kiterjedésű, gyakorlatilag a teljes rendelkezésre álló felületet kitölti, végtelennek tekinthető. A nagyobb kiterjedésű vezetők közötti hosszegységre jutó kapacitás nagyobb, ezáltal a karakterisztikus impedancia kisebb, mint a CPS esetében. Az összehasonlításhoz használt méretek és karakterisztikus impedancia értékek \aref{tab:slotline-cps}. táblázatban láthatóak.
\begin{table}[h!]
	\centering
	\begin{tabular}{||c|c|c|c|c||c|c||}
	\hline
	$\varepsilon_r$ & $t_{sub}$ & $t_{c}$ & $g_{cps}$ & $w_{cps}$ & $Z_{c}$ & $Z_{slotline}$ \\ [0.5ex] 
	\hline\hline
	4,3 & \SI{1,55}{mm} & \SI{0,03}{mm} & \SI{0,274}{mm} & \SI{0,8}{mm} & \SI{100,19}{\ohm} & \SI{74,4}{\ohm}\\
	\hline
	\end{tabular}
	\caption{asd}
	\label{tab:slotline-cps}
\end{table}
\section{A balun transzformátor}
	A struktúrában a balun (balanced-unbalanced) transzformátor feladata a rádió kimenetétől érkező 50 Ohm-os, aszimmetrikus tápvonal és az antennát tápláló szimmetrikus vonal egymáshoz illesztése minimális reflexióval. Költségminimalizálás miatt ezt egy nyomtatott struktúrával tervezem megvalósítani, ami nem használ külön diszkrét felületszerelt alkatrészeket. A rádió felől érkező tápvonalnak (és egyben a balun aszimmetrikus portjának) a paraméterezése fix, a cég által használt radio boardokon egységes koplanáris tápvonal hátoldali földlemezzel (Coplanar Waveguide with Ground, CPW-G, CPW). A tápvonal keresztmetszete \aref{fig:cpw}. ábrán látható.
\begin{figure}[h]
	\centering
	\includegraphics[width=0.6\textwidth]{kep/cpw.pdf}
	\caption{A rádió felől érkező koplanár tápvonal (CPW) keresztmetszete.}
	\label{fig:cpw}
\end{figure}
	A CPW tápvonal alapvetően jó tulajdonságokkal rendelkezik. Csak kis mértékben diszperzív, mert a két vezető között az elektromos erővonalak nagy része az $\epsilon_r$ relatív dielektromos állandójú szubsztrátban záródik, így nincs nagy jelentőssége a levegőben és szubsztrátban terjedő rész-hullámok fázissebességbeli különbségének, viszonylag pontosan leírható QTEM közelítéssel.
\par A balun tervezéséhez sajnos nem találtam olyan forrást, ami a fentebb leírt két tápvonal közötti illesztést valósít meg (CPWG és CPS), így a más esetekre használható megoldásokból igyekeztem összekombinálni egy használható balunt.
\par A gyakorlatban használt nyomtatott balunok egyik nagy csoportja olyan struktúrájú, aminél egyáltalán nincs hátoldali földkitöltés, ennek tipikus példája az MMIC-k (Monolithic Microwave Integrated Circuit) alkalmazási területe, ahol a teljes vezető mintázat egyrétegű és a félvezető chip felszínén helyezkedik el \cite{cpw-ccs, easy-balun, 0-55GHz, 0-110GHz}.
\par A másik struktúracsoport, amihez találtam forrásokat, ugyan egy hátoldali földkitöltéssel rendelkező és egy anélküli tápvonal közötti átmenetet valósít meg, de minden esetben az előbbi tápvonaltípust a Microstrip vonal képviseli. Ezekhez a struktúrákhoz képest a felső oldali földkitöltés megléte jelent külön kihítvást a CPWG tápvonal körül az általam megoldandó problémánál \cite{cpw-ccs, uwb-ms-cps, conv-balun-1, conv-balun-2, wb-cps-ms}.
\section{Az antenna}
	Az antenna tervezésénél az első lépést az jelentette, hogy felderítsem a BIFA struktúrához tartozó, realizálható bemeneti impedancia tartományt, mert várhatóan az antenna bemeneti impedancia a fő korlátozó tényező a struktúra differenciális részének paraméterezésére nézve. Ezt a lépést egy relatíve egyszerű szimulációs modellel végeztem CST-ben, aminek a különböző méretparamétereinek hangolásával értem el a különböző valós bemeneti impedanciákat. Ez a struktúra \aref{subfig:bifa}. ábrán látható.
Ez a lépés lényegében próbálkozásból állt, különböző rögzített karakterisztikus impedanciájú tápvonallal gerjesztve igyekeztem az antenna paramétereinek változtatásával kihangolni azt a kérdéses sávon valós impedanciára. Eredményképpen azt kaptam, hogy körülbelül a \SIrange{100}{130}{\ohm} tartományban gond nélkül kihangolható az antenna. Ebből a tartományból végül a \SI{100}{\ohm}-os bemeneti impedancia mellett döntöttem, mert a balunt kisebb reflexiókkal lehet megvalósítani, ha a be- és kimenete között kisebb mértékű impedanciatranszformációra van szükség.
\par A következőkben elkészítettem többféle BIFA variáció modelljét CST-ben, amiket később a paramétereik beállításával kihangoltam a kiválasztott bemeneti impedanciára. A variációk közül azok, amelyek viszonylag jól kezelhetőek \aref{fig:bifa-variaciok}. ábrán láthatóak.
\begin{figure}[h]
	\centering
	\begin{subfigure}[b]{0.48\textwidth}
		\centering
		\includegraphics[width=\textwidth]{kep/results/bifa_3D.png}
		\caption{egyszerű BIFA}
		\label{subfig:bifa}
	\end{subfigure}
	\hfill
	\begin{subfigure}[b]{0.48\textwidth}
		\centering
		\includegraphics[width=\textwidth]{kep/results/bifa_broadband_3D.png}
		\caption{kiszélesedő, visszahajló szárú BIFA}
	\end{subfigure}
	\hfill
	\begin{subfigure}[b]{0.48\textwidth}
		\centering
		\includegraphics[width=\textwidth]{kep/results/bifa_meandered_3D.png}
		\caption{meanderezett szárú BIFA}
	\end{subfigure}
	\caption{A BIFA szimulált variációi közül a jobban használható változatok.}
	\label{fig:bifa-variaciok}
\end{figure}
