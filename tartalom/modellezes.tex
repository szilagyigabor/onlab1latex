\chapter{Modellezés}
A probléma négy lényeges részből áll, amelyeket többé-kevésbé külön lehet kezeni egymástól:
\begin{enumerate}
	\item Balun
	\item Differenciális vonal
	\item Antenna
	\item Földkitöltésben gerjesztett áramokat csökkentő mintázat
\end{enumerate}
\par Ezek fejlesztése több szempontból is párhuzamosan kellett, hogy történjen, mert sem a balun, sem az antenna nem valósítható meg széles paramétertartományban. Az egyik lényeges korlátozó paraméter a hullámimpedancia -- az antenna bemeneti impedanciáját a balun kimeneti impedanciájához kell illeszteni a kérdéses frekvenciatartományban (\SIrange{2405}{2485}{MHz}) és természetesen ehhez az impedanciához kell méretezni az antennát tápláló differenciális vonalat, hogy ez se okozzon lényeges reflexiót.
\section{Differenciális vonal}
Mivel az antenna differenciális táplálású, érdemes differenciális tápvonallal gerjeszteni. Az antenna által lefedett terület alatt nem célszerű földkitöltést használni az alsóbb rézrétegeken, mert ezek erősen lerontanák az antenna tulajdonságait. Így a tápláló differenciális vonalnak is olyannak kell lennie, hogy nincs alatta földkitöltés. Egy ilyen tápvonal a koplanár differenciális vonal (coplanar strip, CPS). A tápvonal keresztmetszete \aref{fig:cps}. ábrán látható.
\begin{figure}[h]
	\centering
	\includegraphics[width=0.6\textwidth]{kep/cps.pdf}
	\caption{Az antennát tápláló differenciális koplanár vonal (CPS) keresztmetszete.}
	\label{fig:cps}
\end{figure}
\par Sajnos az ilyen tápvonalakban nem lehetséges a TEM hullámterjedés, mivel a vezetők közötti elektromos erővonalak egy része a hordozó anyagában záródik, míg egy másik részük a levegőben, emiatt ugyanannak a haladó hullámnak egyes részei különböző sebességgel haladnak a vonalon, longitudinális térerősség-komponenseket hozva létre. Ez a jelenség pedig diszperzióhoz vezet és megbonyolítja a tápvonal hullámimpedanciájának definícióját is.
\par A CPS relatíve nehezen kezelhető volta miatt gyakorlatilag csak RF IC-ken kialakított differenciális vonalként használják, ha feltétlenül differenciális tápvonalra van szükség, például keverőknél.
\section{A nyomtatott balun transzformátor}
	A struktúrában a balun (balanced-unbalanced) transzformátor feladata a rádió kimenetétől érkező 50 Ohm-os, aszimmetrikus tápvonal és az antennát tápláló differenciális vonal egymáshoz illesztése minimális reflexióval. A rádió felől érkező tápvonalnak (és egyben a balun aszimmetrikus portjának) a paraméterezése fix, a cég által használt radio boardokon egységes koplanáris tápvonal hátoldali földlemezzel (Coplanar Waveguide with Ground, CPW-G, CPW). A tápvonal keresztmetszete \aref{fig:cpw}. ábrán látható.
\begin{figure}[h]
	\centering
	\includegraphics[width=0.6\textwidth]{kep/cpw.pdf}
	\caption{A rádió felől érkező koplanár tápvonal (CPW) keresztmetszete.}
	\label{fig:cpw}
\end{figure}
	A CPW tápvonal alapvetően jótulajdonságokkal rendelkezik. Csak kis mértékben diszperzív, mert a két vezető között az elektromos erővonalak nagy része az $\epsilon_r$ relatív dielektromos állandójú szubsztrátban záródik, így nincs nagy jelentőssége a levegőben és szubsztrátban terjedő rész-hullámok fázissebességbeli különbségének, egyszerűen QTEM közelítéssel kezelhető.
\section{Az antenna}
	Az antenna tervezésénél az első lépést az jelentette, hogy felderítsem a BIFA struktúrához tartozó, realizálható bemeneti impedancia tartományt, mert várhatóan a realizálható antenna bemeneti impedancia a fő korlátozó tényező a struktúra differenciális részének paraméterezésére nézve. Ezt a lépést egy relatíve egyszerű struktúrával végeztem, aminek a különböző méretparamétereinek hangolásával értem el a különböző valós bemeneti impedanciákat a bemeneten. Ez a struktúra \aref{fig:egyszeru.bifa}. ábrán látható.
\par A lépés lényegében próbálkozásból állt, különböző rögzített paraméterek mellett hangoltam ki az antennát a kérdéses sávon valós impedanciára.