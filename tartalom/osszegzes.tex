\chapter{Összegzés}
A félév során megismerkedtem az AWR Microwave Office 3D RF szimulátorával, alap szinten az OpenEMS Matlab/Octave alapú EM szimulátorral és a CST Studio-val. Végül az utóbbiban fejlesztettem egy nyomtatott BIFA típusú antennát sikeresen, valamint egy nyomtatott balun transzformátort részleges sikerrel. A félév elején kitűzött feladatoknak csak egy részét teljesítettem, de irodalomkutatásom során sok olyan anyagra találtam, amik alapján a későbbiekben még folytathatom a munkámat. Ezt a munkahátralékot, vagyis a balun újratervezését és a NYÁK földkitöltésének áramblokkoló mintázatának tervezését, valamint ezek mérését a következő félévben az Önálló Laboratórium 2 c. tárgy keretében tervezem folytatni.